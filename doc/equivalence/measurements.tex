\section{Measurements in the case of the destabilizer tableau}\todo{Dim: I will make this description much more detailed by defining defining measurements for the standard stabilizers, define rowsum, provide examples }	


The tableau extended by the ``destabilizers'' is very similar to the original formulation of the tableau. In the extended version, the number of rows are double because we include two versions of the $X$-matrix and the $Z$-matrix together with their phase factors. In the extended tableau, $X_S$ and $Z_S$ correspond to the usual $X$ and $Z$ matrices,i.e., the stabilizers. The $X_D$ and $Z_D$ which are concatenated at the top of the stabilizers are the de-stabilizers. If the initial state is the $\ket{0}^{\otimes n}$ state, the $X_D$-matrix is initialized with the same values as $Z_S$-matrix. Similarly, the $Z_D$-matrix is initialized with the same values as the $X_S$-matrix.

\begin{equation} 
	\begin{bmatrix}
		X_D & \vert & Z_D & \vert & r_D \\
		\hline
		X_S & \vert & Z_S & \vert & r_S \\
	\end{bmatrix}
\end{equation}

The symplectic inner product of the $i$-th row of the destabilizers and any row $j$ of the stabilizers is equal to 0 unless $i=j+n$. This means that any row will commute with any other row, but it will anti-commute with the row that appears on $n$ rows away. This property remains invariant under the update by Clifford gates (..... to cite Aaronson improved simulation Prop 3.....).


In the original formulation of the tableau, i.e. without the destabilizers, the algorithm for deciding the measurement outcome involves Gaussian elimination which has an overhead of $O(n^3)$. In the case of the tableaus with the destabilizers this is replaced by an operation known as rowsum which reduces this overhead to $O(n^2)$.   

We describe the algorithm for deciding the measurement outcome for a tableau that includes the destabilizers. We assume that the $a$th qubit is the measured qubit. There are two possibilities, either the outcome will be deterministic or it will be probabilistic and the process will involve some coin-flipping. So the first step is to decide whether the outcome is deterministic. We check the stabilizer part of the tableau. If we can find any 1s in the X-matrix, the outcome will be probabilistic. In the case where we can find multiple 1s we pick the one with the smallest index. Say $p$ is the row that corresponds to the measured qubit. Then we update the state as follows:
\begin{enumerate}
	\item call rowsum(i,p) $\forall i\neq p$ such that $x_{ia}=1$
	\item make a coin-flip, suppose it's outcome is $k$, where $k\in\{ -1,1 \}$
	\item set the $(p-n)^{th}$ row to $p$
	\item set the $p^{th}$ row to $z_a$, where $z_a$ is a row with a 1 in it's $a^{th}$ entry 
\end{enumerate} 
The measurement outcome is $r_p=k$\\\\

For the deterministic case:

\begin{enumerate}
	\item Check that the measurement is deterministic, there should be no $p\in \{ n+1 , \cdots , 2n \}$ s.t. $x_{pa}=1$
	\item introduce an additional row to the tableau, the $(2n+1)^{th}$ row, with all it's bits set to 0. When $x_{ia}=1$, perform $\text{rowsum}(2n+1,i+n)$ for all $i\in \{ 1, \cdots , n \}$ 
\end{enumerate}
Then $r_{2n+1}$ will be the outcome of the measurement.\\ 

The rowsum$(h,i)$ is an efficient way to simulation operations between Pauli strings. It is an operation that accepts 4 bits from the tableau, here rowsum$(h,i)$ accepts two entries of the rows $i$ and $h$ respectively. It works as follows, for the inputs $x_t z_t, x_q z_q $ of the rows $i$ and $h$ we get the output $g(x_t z_t, x_q z_q)$, where the function $g$ is defined below.

\begin{equation}
	g(x_t, z_t, x_q, z_q) =
	\begin{cases}
		0 & \text{if } x_t = 0 \text{ and } z_t = 0 \\
		z_q - x_q & \text{if } x_t = 1 \text{ and } z_t = 1 \\
		z_q(2x_q - 1) & \text{if } x_t = 1 \text{ and } z_t = 0 \\
		x_q(2z_q - 1) & \text{if } x_t = 0 \text{ and } z_t = 1 \\
	\end{cases}
\end{equation}

In order to calculate the signs, we use the rowsum operation as a routine and calculate as follows:

\begin{equation}
	r_h=
	\begin{cases}
		0 \text{ if } 2r_h + 2r_i + \sum_{j=1}^{n} g(x_{ij} z_{ij}, x_{hj} z_{hj})= 0 \text{mod}4\\
		1 \text{ if } 2r_h + 2r_i + \sum_{j=1}^{n} g(x_{ij} z_{ij}, x_{hj} z_{hj})= 2 \text{mod}4\\
	\end{cases}
\end{equation}

Note that 0 and 2 are the only possibilities here.	

%To give example here


\begin{example}
	Consider the following tableau,
	
	\begin{equation}
		\begin{bmatrix}
			\hspace{2mm} 1 & \hspace{2mm} 0 \hspace{2mm} & \vline & \hspace{2mm}  1 \hspace{2mm} & \hspace{2mm} 0 \hspace{2mm} & \vline & \hspace{2mm} 0 \hspace{4mm} \\
			\hspace{2mm} 0 & \hspace{2mm} 1 \hspace{2mm} & \vline & \hspace{2mm} 0 \hspace{2mm} & \hspace{2mm} 0 \hspace{2mm} & \vline & \hspace{2mm} 0 \hspace{4mm} \\
			\hline
			\hspace{2mm} 0 & \hspace{2mm} 0 \hspace{2mm} & \vline & \hspace{2mm} 1 \hspace{2mm} & \hspace{2mm} 0 \hspace{2mm} & \vline & \hspace{2mm} 0 \hspace{4mm} \\
			\hspace{2mm} 0 & \hspace{2mm} 0 \hspace{2mm} & \vline & \hspace{2mm} 0 \hspace{2mm} & \hspace{2mm} 1 \hspace{2mm} & \vline & \hspace{2mm} 0 \hspace{4mm} \\
			\hline
			\hspace{2mm} 0 & \hspace{2mm} 0 \hspace{2mm} & \vline & \hspace{2mm} 0 \hspace{2mm} & \hspace{2mm} 0 \hspace{2mm} & \vline & \hspace{2mm} 0 \hspace{4mm} \\
		\end{bmatrix}
	\end{equation}
	
	Here the fifth row, is an extra row which is useful to introduce when performing measurements that fall into the deterministic case. We will perform measurement on the 1st qubit. 
	
	\begin{itemize}
		\item Observe that the rows from $n+1$ to $2n$ contain no $1$'s in the $X$ matrix. %I must be clarify that we still call just the \(X\) matrix the left half of the tableau.
		\item We introduce the $2n+1$ row (with all zeroes), and perform the rowsum operation.
		\begin{itemize}
			\item Calculate $g(x_{51}z_{51},x_{31}z_{31}) = g(0 1 , 0 0 ) = 0$. So, $x_{51}$ must be updated to 0, which is the value that this entry had already. Therefore, it remains unchanged.
		\end{itemize}
		\item Calculating the sign according to the sign formula returns $r_{5} = 0$.
	\end{itemize}
	
	Therefore the outcome of the measurement is 0.
	
\end{example}




\begin{example}
	Consider the following tableau,
	
	\begin{equation}
		\begin{bmatrix}
			\hspace{2mm} 1 & \hspace{2mm} 0 \hspace{2mm} & \vline & \hspace{2mm}  1 \hspace{2mm} & \hspace{2mm} 0 \hspace{2mm} & \vline & \hspace{2mm} 0 \hspace{4mm} \\
			\hspace{2mm} 0 & \hspace{2mm} 1 \hspace{2mm} & \vline & \hspace{2mm} 0 \hspace{2mm} & \hspace{2mm} 0 \hspace{2mm} & \vline & \hspace{2mm} 0 \hspace{4mm} \\
			\hline
			\hspace{2mm} 1 & \hspace{2mm} 0 \hspace{2mm} & \vline & \hspace{2mm} 0 \hspace{2mm} & \hspace{2mm} 0 \hspace{2mm} & \vline & \hspace{2mm} 0 \hspace{4mm} \\
			\hspace{2mm} 0 & \hspace{2mm} 0 \hspace{2mm} & \vline & \hspace{2mm} 0 \hspace{2mm} & \hspace{2mm} 1 \hspace{2mm} & \vline & \hspace{2mm} 0 \hspace{4mm} \\
			
		\end{bmatrix}
	\end{equation}
	
	The measurement will be performed on the 2nd qubit.
	
	
	
	
	
	
	\begin{itemize}
		\item Observe that $x_{31}=1$. Therefore, this example falls into the probabilistic case and the state must be updated.
		\item Only a single 1 appears on the 2nd column, i.e.,  $x_{22}=1$. We calculate $g(x_{22}z_{22},x_{32}z_{32})= g(1 0,0 0 )=0 $.
		\item Make a coin-flipping, suppose it's outcome is $1$.
		\item Set the $1$st row equal to the $3$rd row.
		\item Set the $3$rd row equal to $0$ except for the entry $z_{32}$. 
		
	\end{itemize}
	
	The outcome of the measurement is $r_3=1$, as it was determined by the coin-flipping. The state is updated as:
	
	\begin{equation}
		\begin{bmatrix}
			\hspace{2mm} 1 & \hspace{2mm} 0 \hspace{2mm} & \vline & \hspace{2mm} 0 \hspace{2mm} & \hspace{2mm} 0 \hspace{2mm} & \vline & \hspace{2mm} 0 \hspace{4mm} \\
			\hspace{2mm} 0 & \hspace{2mm} 1 \hspace{2mm} & \vline & \hspace{2mm} 0 \hspace{2mm} & \hspace{2mm} 0 \hspace{2mm} & \vline & \hspace{2mm} 0 \hspace{4mm} \\
			\hline
			\hspace{2mm} 0 & \hspace{2mm} 0 \hspace{2mm} & \vline & \hspace{2mm} 0 \hspace{2mm} & \hspace{2mm} 1 \hspace{2mm} & \vline & \hspace{2mm} 1 \hspace{4mm} \\
			\hspace{2mm} 0 & \hspace{2mm} 0 \hspace{2mm} & \vline & \hspace{2mm} 0 \hspace{2mm} & \hspace{2mm} 1 \hspace{2mm} & \vline & \hspace{2mm} 0 \hspace{4mm} \\
			
		\end{bmatrix}
	\end{equation}
	
	
\end{example}